%% -!TEX root = AUTthesis.tex
% در این فایل، عنوان پایان‌نامه، مشخصات خود، متن تقدیمی‌، ستایش، سپاس‌گزاری و چکیده پایان‌نامه را به فارسی، وارد کنید.
% توجه داشته باشید که جدول حاوی مشخصات پروژه/پایان‌نامه/رساله و همچنین، مشخصات داخل آن، به طور خودکار، درج می‌شود.
%%%%%%%%%%%%%%%%%%%%%%%%%%%%%%%%%%%%
% دانشکده، آموزشکده و یا پژوهشکده  خود را وارد کنید
\faculty{دانشکده مهندسی برق}
% گرایش و گروه آموزشی خود را وارد کنید
\department{گرایش کنترل}
% عنوان پایان‌نامه را وارد کنید
\fatitle{طراحی مسیر و پیاده‌سازی کنترل پلتون  خودروهای  
\\[.75 cm]
 هوشمند در بستر اینترنت اشیاء}
% نام استاد(ان) راهنما را وارد کنید
\firstsupervisor{دکتر حیدرعلی طالبی}
\secondsupervisor{دکتر ایمان شریفی}
% نام استاد(دان) مشاور را وارد کنید. چنانچه استاد مشاور ندارید، دستور پایین را غیرفعال کنید.
\firstadvisor{دکتر حیدرعلی طالبی}
%\secondadvisor{دکتر ایمان شریفی}
% نام نویسنده را وارد کنید
\name{روزبه }
% نام خانوادگی نویسنده را وارد کنید
\surname{بازرگانی}
%%%%%%%%%%%%%%%%%%%%%%%%%%%%%%%%%%
\thesisdate{فروردین 1400}

% چکیده پایان‌نامه را وارد کنید
\fa-abstract{
در این پایان‌نامه کنترل دسته‌ای از ربات‌ها توسط روش‌های سیستم‌های چندعاملی به همراه مسیریابی توسط میدان پتانسیل، الگوریتم‌های ابتکاری همانند $A^*$ و همچنین \lr{Q-learning} که یک روش یادگیری تقویتی است، شبیه‌سازی شده است. در کنترل اجماع روش‌های تک انتگرالی پیاده‌سازی گشته‌اند. به دلیل تاخیر در رسیدن اطلاعات عامل رهبر به بقیه ربات‌ها، از رهبر مجازی استفاده شده است. یک روش برای کنترل اجماع پیشنهاد گردیده و پایداری آن به کمک پایداری لیاپانوف اثبات گشته است. برای مسیریابی توسط الگوریتم \lr{Q-learning} توابع پاداش مختلف بررسی و تحلیل گشته‌اند. در نهایت، دسته ربات‌ها با الگوریتم‌های مسیریابی متفاوت، در محیطی از موانع ثابت و متحرک، مورد امتحان واقع شدند.
}


% کلمات کلیدی پایان‌نامه را وارد کنید
\keywords{سیستم‌های چندعاملی، کنترل اجماع، هوش مصنوعی، روش‌های ابتکاری، یادگیری تقویتی، \lr{Q-learning}, الگوریتم $A^*$}



\AUTtitle
%%%%%%%%%%%%%%%%%%%%%%%%%%%%%%%%%%
\vspace*{7cm}
\thispagestyle{empty}
\begin{center}
\includegraphics[height=5cm,width=12cm]{besm}
\end{center}