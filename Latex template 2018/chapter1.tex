\chapter{مقدمه}

\section{مقدمه}

در این فصل ابتدا به بیان هدف پروژه و انگیزه از طرح آن پرداخته می‌شود. سپس دلیل نیاز و فواید سیستم‌های چندعاملی پرداخته خواهد شد. در ادامه به تاثیر هوش مصنوعی در رباتیک اشاره خواهد شد و در انتها شمای کلی از فصل‌های پایان‌نامه ارائه می‌شوند.

\section{هدف و انگیزه}
امروزه با توسعه هوشمندسازی، انواع مختلف ربات‌ها با توانمندی‌های متنوع مورد استفاده قرار می‌گیرند. عملکرد گروه ربات‌ها و تعامل بین آن‌ها در صنایع مختلف کاربرد دارد. که یکی از انواع آن، سیستم چند عاملی\LTRfootnote{multi-agent systems} و به طور خاص اجماع\LTRfootnote{platoon} است. منظور از اجماع دسته‌ای از ربات‌هاست که به هدایت یکی از آن‌ها، در کنار یکدیگر با نظم و ترتیب، فاصله استاندارد و با سرعت بهینه در کنار یکدیگر حرکت کنند.

در این پروژه هدف کنترل ربات‌های چرخ‌دار متحرک\LTRfootnote{mobile wheeled robots}، مدل دیفرانسیلی\LTRfootnote{differential model}، نیمه خودران و پیاده‌سازی اجماع ربات‌ها است که این سیستم به کمک شبیه‌ساز\LTRfootnote{Simulink} متلب\LTRfootnote{MATLAB} و برای نمایش به کمک ابزار شبیه سازی ربات‌های متحرک\LTRfootnote{mobile robotics simulation toolbox} پیاده سازی گردیده است.

با بهره‌گیری از فناوری اینترنت اشیاء\LTRfootnote{Internet of Things} می‌توان ارتباط ربات‌ها را در بستر اینترنت به یکدیگر و یا به مراکز مختلف میسر نمود. به این ترتیب می‌توان به کمک آن برای ربات‌ها مسیر طراحی‌ و ترافیک مسیر و اجماع ربات‌ها را تا رسیدن به مقصد کنترل کرد.

از مزایای اجماع عبارت از تسریع در عملیات‌ و کاهش انرژی مصرفی و استهلاک ربات‌ها است. از جمله کاربردهای پلتون می‌توان به خودروهای سنگین نیمه هوشمند بین شهری، در جاده‌های اختصاصی شرکت‌ها برای جابه‌جایی مواد اولیه و فراورده‌ها و درون انبارها اشاره کرد. همچنین در جابه‌جایی اجسامی که یک ربات به تنهایی از انجام آن ناتوان است، در عملیات‌های امدادی جست و جو و حتی در فضا پلتون استفاده می‌گردد که نشان از اهمیت بسیار بالای آن است.

امروزه علم مدیریت ترافیک با ترکیب تکنولوژی‌های پیشرفته با زیر ساخت‌های شهری، بسیار نوین گشته است. این امر باعث جذب علاقه مدیران برای استفاده از سازمان حمل و نقل هوشمند\LTRfootnote{Intelligent Transportation System} شده است \cite{baskar2011traffic}. رکن اصلی سازمان حمل و نقل هوشمند خودرو هوشمند\LTRfootnote{intelligent vehicle} است. خودروهای هوشمند دارای ویژگی‌هایی هستند که از مهمترین آن‌ها می‌توان به خودران\LTRfootnote{autonomous vehicles} بودن خودروها اشاره کرد.

خودران بودن خودروها علاوه بر اینکه راحتی و وقت بیشتر به خاطر آزاد بودن سرنشینان را به همراه دارد، می‌تواند از اشتباه‌های رانندگان خودروها نیز جلوگیری و به راحتی مدیریت کرد. در نتیجه ترافیک کمتر و امنیت بیشتری را در سطح جاده‌ها فراهم آورد.



















